\chapter{Реализация алгоритма}
\section{Предлагаемая схема параллелизации}
Две основные процедуры в алгоритме Фортена с модификацией Буздалова --- $NDHelperA$ и $NDHelperB$. Процедура $NDHelperA$ делит задачу на подзадачи и сливает результаты воедино посредством $NDHelperB$. Процедура $NDHelperB$ сама по себе является рекурсивной, следующей парадигме <<разделяй и властвуй>>. Деление задач происходит до момента, когда размерность $k$ станет равна $2$, что в дальнейшем обрабатывается отдельными процедурами.

Можно заметить, что на момент вызова процедуры $NDHelperB(L, H, k)$ ранги точек из множества $L$ уже вычислены и в дальнейшем меняться не будут. Также заметим, что порядок вызова подзадач в теле $NDHelperB$ не нарушит корректности алгоритма. Более того, мы можем разделить множество $L$ на любое количество частей ${L_1, L_2,\ldots, L_n}$ и тогда результат исполнения $NDHelperB(L_i, H, k)$ для всех $i$ будет идентичен результату исполнения $NDHelperB(L, H, k)$. Таким образом, $NDHelperB$ допускает параллельное исполнение.

С параллелизацией процедуры $NDHelperA$ возникают сложности. Внутренние вызовы $NDHelperA$ и $NDHelperB$ зависят друг от друга и имеют строго определенную последовательность. 

\section{Детали реализации}
Алгоритм был реализован на языке программирования Java с использованием Fork/Join фреймворка, который хорошо подходит для распараллеливания рекурсивных задач.

