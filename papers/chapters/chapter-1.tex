\chapter{Обзор предметной области}
В этой главе описываются основные понятия и термины предметной области, к которой относится представленная работа. Также проводится обзор имеющихся алгоритмических решений и формулируется постановка задачи.

\section{Основные определения}

Принципиальное отличие многокритериальных задач оптимизации от однокритериальных заключается в том, что во втором случае целью является поиск самого оптимального решения. В случае же задачи многокритериальной оптимизации такого решения может не существовать вследствие возможных конфликтов целевых функций. Таким образом, многокритериальная оптимизация основывается на компромиссном поиске группы оптимальных решений в смысле Парето.
\begin{definition}
    В $M$-мерном пространстве, точка $A = (a_1, \ldots, a_M)$ \textit{доминирует в смысле Парето} точку $B = (b_1, \ldots, b_M)$, когда для всех $1 \leq i \leq M$ выполняется неравенство $a_i \leq b_i$ и существует хотя бы одно такое $j$, что $a_j < b_j$.
\end{definition}
\begin{definition}
    Множество оптимальных по Парето недоминируемых решений называется \textit{Парето-фронтом}.
\end{definition}
\begin{definition}
    \textit{Недоминирующая сортировка} множества точек $S$ в $M$-мерном пространстве --- это процедура, в процессе которой всем точкам, которые не доминируются никакими другими точками, назначается ранг $0$. Далее всем точкам, которые доминируются только точками с рангом $0$, назначается ранг $1$, и т.д. Все точки с рангом $i$ доминируются только точками с рангом не более $i - 1$.
\end{definition}

\section{Обзор существующих алгоритмов}
\subsection{Тривиальный алгоритм}
Рассмотрим самую наивную версию алгоритма недоминирующей сортировки. Найдем все недоминируемые точки с нулевым рангом, вычеркнем эти точки и будем повторять эту процедуру, каждый раз назначая ранг на единицу больше, чем на предыдущей итерации. Тогда, если $M$ --- количество критериев отбора или размерность множества точек (решений), а $N$ --- количество решений, нам необходимо за время $O(MN^2)$ сравнить $O(N^2)$ пар точек по каждому из $M$ критериев и повторить эту процедуру в худшем случае $N$ раз. Таким образом получаем время работы $O(MN^3)$.

Кунг и др.~\cite{kung75} в своей работе предложили алгоритм поиска недоминируемых решений со сложностью $O(N\log^{M-1}N)$. Совместив предложенный алгоритм с идеей удаления найденных точек, описанной в наивном алгоритме, мы получаем алгоритм недоминирующей сортировки с общей сложностью $O(N^2\log^{M-1}N)$ в худшем случае, если максимальный ранг решений --- $O(N)$.

\subsection{Быстрая недоминирующая сортировка}
Деб~\cite{deb00} в своей работе предложил алгоритм <<Быстрой недоминирующей сортировки>> как часть алгоритма NSGA-II, улучшив асимптотическую сложность алгоритма до $O(MN^2)$. Этот алгоритм является базисным с точки зрения эффективности в семействе алгоритмов <<Разделяй и властвуй>>.

Йенсен~\cite{jensen03} был первым, кто предложил алгоритм недоминирующей сортировки со сложностью $O(N\log^{M-1}N)$, позволив тем самым эффективно вычислять ранги точек для типичных конфигураций входных значений. Однако его алгоритм имел существенный недостаток: он работал корректно только в предположении, что никакие два решения не имеют одинаковых значений по одному и тому же критерию. Фанг и др.~\cite{fang08} в своей работе пришли к заключению, что алгоритм Йенсена неспособен генерировать такие же недоминирующие фронты точек, как оригинальный алгоритм NSGA-II. Также они продемонстрировали, что устранение главного недостатка алгоритма Йенсена является нетривиальной задачей.

Решение данной проблемы было предложено Фортеном и др.~\cite{fortin13}. Предложенный им <<обобщенный>> алгоритм недоминирующей сортировки работает корректно на любых входных данных, сохраняя при этом среднюю временную сложность $O(N\log^{M-1}N)$. Тем не менее было доказано, что в худшем случае время работы алгоритма составляет $O(N^2M)$.

Дальнейшее улучшение недоминирующей сортировки Фортена было предложено Буздаловым и Шалыто~\cite{buzdalov14}. В своей работе они модифицируют алгоритм и доказывают оценку $O(N\log^{M-1}N)$ для худшего случая. Более подробное описание алгоритма с модификацией будет представлено ниже.

\subsection{Алгоритм Роя}
Существует также алгоритм <<Best Order Sort>>, предложенный Роем и др.~\cite{roy16} с вычислительной сложностью $O(MN\log{M}+MN^2)$. В худшем случае время его работы --- $O(MN^2)$, но на некоторых входных данных алгоритм может работать за $O(MN\log{M})$. Авторы не предоставляют более строгого теоретического доказательства эффективности данного алгоритма. Как показывает практика, алгоритм демонстрирует не самые лучшие результаты на больших массивах входных данных. В рамках представленной работы данный алгоритм интереса не представляет.

\section{Описание алгоритма <<Разделяй и властвуй>>}
Алгоритм Фортена состоит из нескольких процедур:
\begin{itemize}
    \item $NonDominatedSort(S, K)$ --- главная процедура, получающая на вход множество точек $S$ и размерность $K$ и возвращающая результат недоминирующей сортировки --- последовательность Парето-фронтов, содержащих точки исходного множества. Внутри процедуры осуществляется предварительная обработка данных и вызывается процедура $NDHelperA$.
    \item $NDHelperA(S, k)$ вычисляет ранги точек множества $S$, основываясь на первых $k$ критериях. Процедура может рекурсивно вызывать саму себя, а также $NDHelperB$, $SweepA$ и $SplitA$.
    \item $NDHelperB(L, H, k)$ назначает ранги точкам из множества $H$, сравнивая их с точками из множества $L$, основываясь на первых $k$ критериях. Эта процедура может рекурсивно вызывать саму себя, $SweepB$ и $SplitB$.
    \item $SweepA(S)$ --- это процедура, определяющая ранги точек по первым двум координатам, используя метод заметающей прямой. Время ее работы --- $O(|S|\log|S|)$.
    \item $SweepB(L, H)$ определяет ранги точек из множества $H$, сравнивая их с точками множества $L$ по первым двум критериям. В ней также используется метод заметающей прямой, а время ее работы оценивается как $O((|L| + |H|)\log{|L|})$.
    \item $SplitA(S, k)$ разделяет множество точек $S$ на два подмножества, сравнивая $k$-ю координату каждой точки с медианным значением. Точки, имеющие по $k$-му критерию значение, равное медиане, добавляются к меньшему из двух подмножеств.
    \item $SplitB(L, H, k)$ разделяет каждое из множеств $L$ и $H$ на два, используя аналогичную процедуру. В качестве опорного элемента используется медиана $k$-х координат большего из двух исходных множеств.
\end{itemize}
Стоит отметить, что все вышеперечисленные процедуры сохраняют лексикографический порядок исходных множеств.

Буздалов и др. предложили модифицировать процедуры $SplitA$ и $SplitB$, а также соответственно вызывающие их процедуры $NDHelperA$ и $NDHelperB$. Вместо первых двух будет использоваться метод $SplitBy(S, m, k)$, который разделяет входящее множество точек $S$ не на две, а на три части:
\begin{itemize}
    \item $L$ --- множество точек, $k$-я координата которых меньше $m$;
    \item $M$ --- множество точек, $k$-я координата которых равна $m$;
    \item $H$ --- множество точек, $k$-я координата которых больше $m$;
\end{itemize}

\begin{algorithm}[!h]
\caption{Процедура SplitBy. Разделение точек из $S$ на три подмножества по $k$-му критерию относительно значения $m$.}\label{lst0}
\begin{algorithmic}
\Procedure{SplitBy}{S, m, k}
    \State $L \gets \{s \in S\mid s_k < m\}$
    \State $M \gets \{s \in S\mid s_k = m\}$
    \State $H \gets \{s \in S\mid s_k > m\}$
\EndProcedure
\end{algorithmic}
\end{algorithm}

\begin{algorithm}[!h]
\caption{Процедура NDHelperA. Определение рангов точек из $S$ по $k$ первым критериям.}\label{lst1}
\begin{algorithmic}
\Procedure{NDHelperA}{S, k}
    \If{$|S| < 2$}
        \State \Return
    \ElsIf{$|S| = 2$}
        \State $\{s^{(1)}, s^{(2)}\}\gets S$
        \If{$s_{1:k}^{(1)} \prec s_{1:k}^{(2)}$}
            \State $RANK(S^{(2)})\gets \max\{RANK(S^{(2)}), RANK(S^{(1)})+1\}$ 
        \EndIf
    \ElsIf{$k = 2$}
        \State$\textsc{SweepA}(S)$
    \ElsIf{$|\{s_k \mid s \in S\}| = 1$}
        \State $\textsc{NDHelperA}(S, k-1)$
    \Else
        \State $L,M,H \gets \textsc{SplitBy}(S, median\{s_k\mid s \in S\}, k)$
        \State $\textsc{NDHelperA}(L, k)$
        \State $\textsc{NDHelperB}(L, M, k-1)$
        \State $\textsc{NDHelperA}(M, k-1)$
        \State $\textsc{NDHelperB}(L \cup M, H, k-1)$
        \State $\textsc{NDHelperA}(H, k)$
    \EndIf
\EndProcedure
\end{algorithmic}
\end{algorithm}

\begin{algorithm}[!h]
\caption{Процедура NDHelperB. Назначение рангов точкам из $H$ относительно точек из $L$ по $k$ первым критериям.}\label{lst2}
\begin{algorithmic}
\Procedure{NDHelperB}{L, H, k}
    \If{$|L| = 0$ or $|H| = 0$}
        \State \Return
    \ElsIf{$|L| = 1$ or $|H| = 1$}
        \ForAll{$h \in H, l \in L$}
            \If{$l_{1:k} \preceq h_{1:k}$}
                \State $RANK(h) \gets \max\{RANK(h), RANK(l) + 1\}$
            \EndIf
        \EndFor
    \ElsIf{$k = 2$}
        \State $\textsc{SweepB}(L, H)$
    \ElsIf{$ \max\{l_k\mid l \in L\} \leq \min\{h_k\mid h \in H\}$}
        \State $\textsc{NDHelperB}(L, H, k-1)$
    \Else
        \State $m \gets median\{s_k\mid s \in L \cup H\}$
        \State $L_1, M_1, H_1 \gets \textsc{SplitBy}(L, m, k)$
        \State $L_2, M_2, H_2 \gets \textsc{SplitBy}(H, m, k)$
        \State $\textsc{NDHelperB}(L_1, L_2, k)$
        \State $\textsc{NDHelperB}(L_1, M_2, k-1)$
        \State $\textsc{NDHelperB}(M_1, M_2, k-1)$
        \State $\textsc{NDHelperB}(L_1 \cup M_1, H_2, k-1)$
        \State $\textsc{NDHelperB}(H_1, H_2, k)$
    \EndIf
\EndProcedure
\end{algorithmic}
\end{algorithm}


\section{Известные параллельные решения}
В своей работе Гупта и Тэн~\cite{gupta15} представили масштабируемую реализацию алгоритма недоминирующей сортировки, заточенную под многоядерные графические процессоры. Однако их разработка имеет некоторые недостатки. В качестве основания исследователи выбрали стандартную реализацию NSGA-II с квадратичным временем работы. Несмотря на хорошую масштабируемость, алгоритм сильно проигрывает по общему объему работы. Существенный прирост в производительности в их разработке достигается только при очень большом количестве процессоров. 

\section{Постановка задачи}
Как было сказано ранее, разработка эффективных алгоритмов недоминирующей сортировки является актуальной задачей в области эволюционных алгоритмов. Алгоритм Фортена в модификации Буздалова обладает хорошей асимптотической сложностью и показывает хорошие результаты на больших объемах входных данных. Тем не менее, этот алгоритм не использует возможностей многопроцессорных систем, в связи с чем возникает задача его параллелизации.
