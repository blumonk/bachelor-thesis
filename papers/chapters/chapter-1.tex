\chapter{Обзор предметной области}
В этой главе описываются основные понятия и термины предметной области, к которой относится представленная работа. Также проводится обзор имеющихся алгоритмических решений и формулируется постановка задачи.

\section{Основные определения}

Принципиальное отличие многокритериальных задач оптимизации от однокритериальных заключается в том, что во втором случае целью является поиск самого оптимального решения. В случае же задачи многокритериальной оптимизации такого решения может не существовать вследствие возможных конфликтов целевых функций. Таким образом, многокритериальная оптимизация основывается на компромиссном поиске группы оптимальных решений в смысле Парето.
\begin{definition}
    В $M$-мерном пространстве, точка $A = (a_1, \ldots, a_M)$ доминирует в смысле Парето точку $B = (b_1, \ldots, b_M)$, когда для всех $1 \leq i \leq M$ выполняется неравенство $a_i \leq b_i$ и существует хотя бы одно такое $j$, что $a_j < b_j$.
\end{definition}
\begin{definition}
    Недоминирующая сортировка множества точек $S$ в $M$-мерном пространстве --- это процедура, в процессе которой всем точкам, которые не доминируются никакими другими точками, назначается ранг $0$. Далее всем точкам, которые доминируются только точками с рангом $0$, назначается ранг $1$, и т.д. Все точки с рангом $i$ доминируются только точками с рангом не более $i + 1$.
\end{definition}

\section{Обзор существующих алгоритмов}
\subsection{Тривиальный алгоритм}
Рассмотрим самую наивную версию алгоритма недоминирующей сортировки. Найдем все недоминируемые точки с нулевым рангом, вычеркнем эти точки и будем повторять эту процедуру, каждый раз назначая ранг на единицу больше, чем на предыдущей итерации. Тогда, если $M$ --- количество критериев отбора или размерность множества точек (решений), а $N$ --- количество решений, нам необходимо за время $O(MN^2)$ сравнить $O(N^2)$ пар точек по каждому из $M$ критериев и повторить эту процедуру в худшем случае $N$ раз. Таким образом получаем время работы $O(MN^3)$.

Кунг и др. в своей работе предложили алгоритм поиска недоминируемых решений со сложностью $O(N\log^{M-1}N)$. Совместив предложенный алгоритм с идеей удаления найденных точек, описанной в наивном алгоритме, мы получаем алгоритм недоминирующей сортировки с общей сложностью $O(N^2\log^{M-1}N)$ в худшем случае, если максимальный ранг решений --- $O(N)$.

\subsection{Быстрая недоминирующая сортировка}
Деб в своей работе предложил алгоритм <<Быстрой недоминирующей сортировки>>, улучшив асимптотическую сложность алгоритма до $O(MN^2)$. Этот алгоритм является базисным с точки зрения эффективности в семействе алгоритмов <<Разделяй и властвуй>>.

Йенсен был первым, кто предложил алгоритм недоминирующей сортировки со сложностью $O(N\log^{M-1}N)$, позволив тем самым эффективно вычислять ранги точек для типичных конфигураций входных значений. Однако его алгоритм имел существенный недостаток: он работал корректно только в предположении, что никакие два решения не имеют одинаковых значений по одному и тому же критерию. Фанг и др. в своей работе пришли к заключению, что алгоритм Йенсена неспособен генерировать такие же недоминирующие фронты точек, как оригинальный алгоритм NSGA-II. Также они продемонстрировали, что устранение главного недостатка алгоритма Йенсена является нетривиальной задачей.

Решение данной проблемы было предложено Фортеном и др. Предложенный им <<обобщенный>> алгоритм недоминирующей сортировки работает корректно на любых входных данных, сохраняя при этом среднюю временную сложность $O(N\log^{M-1}N)$. Тем не менее было доказано, что в худшем случае время работы алгоритма составляет $O(N^2M)$.

Дальнейшее улучшение недоминирующей сортировки Фортена было предложено Буздаловым и др. В своей работе он модифицирует алгоритм и доказывает оценку $O(N\log^{M-1}N)$ для худшего случая.

\section{Описание алгоритма <<Разделяй и властвуй>>}
