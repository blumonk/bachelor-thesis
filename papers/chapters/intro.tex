\startprefacepage
Эволюционные алгоритмы на сегодняшний день являются одним из cамых популярных способов решения задач многокритериальной оптимизации, в которых необходимо оптимизировать не одну, а несколько конфликтующих целевых функций.

В качестве критерия отбора группы оптимальных решений в задачах многокритериальной оптимизации как правило используется оптимальность в смысле Парето, т.е. осуществляется поиск набора недоминирующих решений.
Таким образом, одним из ключевых шагов в различных многокритериальных эволюционных алгоритмах является процедура недоминирующей сортировки.

Время работы данного шага как правило является определяющим в общей временной сложности вышеупомянутых алгоритмов, поэтому оптимизация процедуры недоминирующей сортировки является ключом к более эффективным решениям задач многокритериальной оптимизации.

За последние несколько десятилетий исследователями было представлено целое множество различных алгоритмов недоминирующей сортировки.
Все эти алгоритмы имеют свои плюсы и минусы, и их эффективность так или иначе зависит от специфики входных данных.

В данной работе представлена параллельная версия алгоритма быстрой недоминирующей сортировки, а также продемонстрированы экспериментальные результаты, доказывающие его эффективность.
