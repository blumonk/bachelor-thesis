\startprefacepage
Эволюционные алгоритмы являются одним из cамых популярных способов решения задач многокритериальной оптимизации на сегодняшний день.

В подобных задачах необходимо оптимизировать не одну, а сразу несколько конфликтующих целевых функций, и такие задачи очень часто встречаются в огромном числе различных областей, начиная от экономики и финансов и заканчивая промышленным дизайном.

В качестве компромиссного критерия отбора группы оптимальных решений в задачах многокритериальной оптимизации, как правило, используется оптимальность в смысле Парето, т.е. осуществляется поиск набора недоминируемых решений.

Таким образом, одним из ключевых шагов в различных многокритериальных эволюционных алгоритмах является процедура недоминирующей сортировки, решающая эту задачу.

Время работы данного шага, как правило, является определяющим в общей временной сложности вышеупомянутых алгоритмов, поэтому разработка эффективных алгоритмов недоминирующей сортировки является ключом к более плодотворному решению задач многокритериальной оптимизации и остается крайне актуальной задачей.

За последние несколько десятилетий исследователями было представлено целое множество различных алгоритмов недоминирующей сортировки.
Все эти алгоритмы имеют свои плюсы и минусы, и их производительность так или иначе зависит от специфики входных данных.

Целью данной работы является разработка эффективного параллельного алгоритма недоминирующей сортировки.

В первой главе будет приведен краткий обзор предметной области и даны определения ключевых понятий.
Также в этой главе будут описаны самые популярные и актуальные на сегодняшний день алгоритмы недоминирующей сортировки, в том числе известные на текущий момент параллельные разработки.

Во второй главе будет предложена схема преобразования алгоритма Фортена в модификации Буздалова, позволяющая улучшить производительность сортировки благодаря параллельному исполнению.
Для исследования был выбран именно этот алгоритм, потому что он обладает наилучшим доказанным теоретическим временем работы из всех известных в настоящее время алгоритмов недоминирующей сортировки.

В третьей главе будет представлено описание реализации предложенного алгоритма, а также результаты экспериментального исследования его эффективности в сравнении с оригинальной версией и другими параллельными решениями.

В заключении будут подведены итоги проделанной работы, а также описаны дальнейшие планы и перспективы улучшения предложенного алгоритма.
